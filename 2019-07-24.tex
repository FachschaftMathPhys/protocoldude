% !TEX program    = pdflatex
% !TEX encoding   = UTF-8
% !TEX spellcheck = de_DE

\documentclass[11pt, fachschaft=mathphys,twosided=true]{mathphys/mathphys-article}
\usepackage[utf8]{inputenc}
\usepackage[ngerman]{babel}
\usepackage[T1]{fontenc}
\usepackage{eurosym}
\usepackage{booktabs}
\renewcommand\thesection{TOP \arabic{section}:}
\renewcommand*\thesubsection{TOP \arabic{section}.\arabic{subsection}:}
\renewcommand\contentsname{Tagesordnung}
\newenvironment{antrag}{\begin{quote}\begin{itshape}}{\end{itshape}\end{quote}}
\usepackage{hyperref}
%-------------------------------------------------
% Konsensvorlagen (ggf. anpassen!)
%-------------------------------------------------
\newcommand{\konsens}[1]{In der Fachschaftssitzung MathPhysInfo, sowie in den anwesenden Fachschaftsräten, besteht Konsens ohne Bedenken.\\} % immer die Anzahl der Anwesenden anpassen!
\newcommand{\konsensLB}[1]{In der Fachschaftssitzung MathPhysInfo, sowie in den anwesenden Fachschaftsräten, besteht Konsens mit leichten Bedenken.\\} % immer die Anzahl der Anwesenden anpassen!
\newcommand{\konsensE}[1]{In der Fachschaftssitzung MathPhysInfo, sowie in den anwesenden Fachschaftsräten, besteht Konsens mit Enthaltung.\\} % immer die Anzahl der Anwesenden anpassen!
\newcommand{\konsensFsrPhys}{Die Fachschaftsratssitzung Physik entscheidet einstimmig, den Beschluss entsprechend der Entscheidung der Fachschaftssitzung MathPhysInfo umzusetzen.\\}
% \newcommand{\konsensFsrMathe}{Die Fachschaftsratssitzung Mathematik entscheidet einstimmig, den Beschluss entsprechend der Entscheidung der Fachschaftssitzung MathPhysInfo umzusetzen.\\}
\newcommand{\konsensFsrInfo}{Die Fachschaftsratssitzung Informatik entscheidet einstimmig, den Beschluss entsprechend der Entscheidung der Fachschaftssitzung MathPhysInfo umzusetzen.\\}

\setlength{\parindent}{0pt}
\setlength{\parskip}{1em}

\begin{document}
\date{\vspace{-2em} 24. Juli 2019 \vspace{-1em}} % Datum ersetzen
\title{\vspace{-2em}Protokoll der Fachschaftssitzung MathPhysInfo}
\maketitle

\begin{tabbing}
    \textbf{Sitzungsmoderation:}\quad\=Kai-Uwe \\% SiMo einfügen
    \textbf{Protokoll:}\> Max Müller \\% Protokoll einfügen
    \textbf{Beginn:}\>18:15 Uhr\\
    \textbf{Ende:}\>xx:xx Uhr\\ % Sitzungsende einfügen
\end{tabbing}

\section{Begrüßung}
    Die Sitzungsmoderation begrüßt die anwesenden Mitglieder der Studienfachschaften Mathematik, Physik und Informatik und eröffnet so die Fachschaftsvollversammlung der Studienfachschaften Mathematik, Physik und Informatik.

\section{Feststellung der Beschlussfähigkeiten}
    Fachschaftsrat Physik, Mathe und Informatik sind alle Beschlussfähig.

\section{Beschluss des Protokolls der letzten Sitzung}

\begin{antrag}
	Annahme des Protokolls vom xx. Monat 2019. \\% Datum einfügen
\end{antrag}
\konsensE{}

\section{Feststellen der Tagesordnung}
\begin{antrag}
    Die Tagesordnung wird in der vorliegenden Form angenommen.
\end{antrag}
\konsens{}

\section{Sitzungsmoderation für die nächste Sitzung}
    Die Sitzungsmoderation für die Fachschaftssitzung MathPhysInfo der nächsten Woche wird von xxx übernommen. % SiMo nachste Woche einfugen

\section{Zukunft des IMP e.V.}

\section{Finanzbeschluss Informatik-Aufnahmeprüfungsgespräche}

\section{Finanzbeschluss ZaPF}

\section{Studium \& Lehre}

\section{Verschiedenes}



\emph{Die Sitzungmoderation schließt die Sitzung um xx:xx Uhr.}
\end{document}
